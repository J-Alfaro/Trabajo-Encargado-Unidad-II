% Generated by GrindEQ Word-to-LaTeX 
\documentclass{article} %%% use \documentstyle for old LaTeX compilers

\usepackage[english]{babel} %%% 'french', 'german', 'spanish', 'danish', etc.
\usepackage{amssymb}
\usepackage{amsmath}
\usepackage{txfonts}
\usepackage{mathdots}
\usepackage[classicReIm]{kpfonts}
\usepackage[dvips]{graphicx} %%% use 'pdftex' instead of 'dvips' for PDF output

% You can include more LaTeX packages here 


\begin{document}

%\selectlanguage{english} %%% remove comment delimiter ('%') and select language if required


\noindent 

\noindent 

\noindent 

\noindent 

\noindent \includegraphics*[width=0.59in, height=0.56in, keepaspectratio=false]{Imagen11}\includegraphics*[width=0.51in, height=0.66in, keepaspectratio=false]{Imagen22}\textbf{ }

\noindent 

\noindent 
\[1\] 


\noindent \includegraphics*[width=1.08in, height=1.46in, keepaspectratio=false]{Imagen1}\textbf{\underbar{}}

\noindent \textbf{}

\noindent \textbf{UNIVERSIDAD PRIVADA DE TACNA}

\noindent \textbf{}

\noindent \textbf{FACULTAD DE INGENIERIA}

\noindent \textbf{\textit{}}

\noindent \textbf{\textit{}}

\noindent \textbf{Escuela Profesional de Ingenier\'{i}a de Sistemas}

\noindent \textbf{}

\noindent 

\noindent \textbf{ INFORME DE LABORATORIO}

\noindent \textbf{``COMPARATIVA DE RAPIDEZ Y SIMPLICIDAD DE USO DE UN DATAWAREHOUSE Y UN DATALAKE''}

\noindent \textbf{}

\noindent Curso: Inteligencia de Negocio

\noindent \textbf{\textit{}}

\noindent \textbf{}

\noindent \textbf{}

\noindent Docente: Ing. ..

\noindent \textbf{}

\noindent \textbf{Aguilar Atencio, Jhon Peter (2015053222)}

\noindent \textbf{Aguilar Atencio, Jhon Peter (2015053222)}

\noindent \textbf{COAQUIRA COAQUIRA, Guimer Senon (2015053226)}

\noindent \textbf{Perez Mamani, Nilton Edy (2015053222)}

\noindent \textbf{}

\noindent \textbf{Tacna -- Per\'{u}}
\[2019\] 
\textbf{\underbar{}}

\noindent \textbf{\underbar{}}

\noindent \textbf{}

\noindent \textbf{}

\noindent \textbf{INDICE}

\noindent 

\noindent RESUMEN 3ABSTRACT 41. INTRODUCCI\'{O}N 52. MATERIALES Y METODOS 63. MARCO TEORICO 63.1. DATA WAREHOUSE 63.2. DATA LAKE 83.3. DIFERENCIAS ENTRE DATA WAREHOUSE Y DATA LAKE 104. CONCLUSIONES 135. RECOMENDACIONES 146. BIBLIOGRAF\'{I}A 15

\noindent \textbf{\underbar{}}

\noindent \textbf{\underbar{}}

\noindent \textbf{\underbar{}}

\noindent \textbf{\underbar{}}

\noindent \textbf{\underbar{}}

\noindent \textbf{\underbar{}}

\noindent \textbf{\underbar{}}

\noindent \textbf{\underbar{}}

\noindent \textbf{\underbar{}}

\noindent \textbf{\underbar{}}

\noindent \textbf{\underbar{}}

\noindent \textbf{\underbar{}}

\noindent \textbf{}

\noindent \textbf{}

\noindent \textbf{}

\noindent \textbf{\underbar{COMPARATIVA DE RAPIDEZ Y SIMPLICIDAD DE USO DE UN DATAWAREHOUSE Y UN DATALAKE}\textit{\underbar{}}}

\noindent \textbf{}

\noindent \textbf{RESUMEN}

\noindent 

\noindent Un sistema de inteligencia de negocio est\'{a} formado por diferentes elementos, pero de todas las piezas la principal de ellas es el datawarehouse o almac\'{e}n de datos, ya que es el componente que almacena los datos a analizar. En este cap\'{i}tulo vamos a explicar qu\'{e} es un data warehouse, c\'{o}mo se cargan los datos en el datawarehouse mediante los procesos de Extracci\'{o}n, Transformaci\'{o}n y Carga (ETL) y c\'{o}mo se pueden consultar los datos de la data warehouse de forma f\'{a}cil, eficiente y potente mediante tecnolog\'{i}as OLAP.

\noindent 

\noindent Seguramente han escuchado muchas veces el t\'{e}rmino de Data Warehouse; podemos definirla como una base de datos corporativa donde se integra y depura informaci\'{o}n de una o varias fuentes distintas, que luego ser\'{a}n procesadas y analizadas desde distintos puntos de vista con afinidad de perspectivas y grandes velocidades de respuesta.

\noindent 

\noindent La creaci\'{o}n del Data Warehouse representa la mayor\'{i}a de las veces el primer paso, desde el punto de vista t\'{e}cnico, para implantar una soluci\'{o}n completa y fiable de Business Intelligence y as\'{i} aportar las mejores respuestas a los problemas de la organizaci\'{o}n.

\noindent 

\noindent Por otro lado, vemos al Datalake es un repositorio de almacenamiento que contienen una gran cantidad de datos en bruto y que se mantienen all\'{i} hasta que sea necesario. A diferencia de un data warehouse jer\'{a}rquico que almacena datos en ficheros o carpetas, un datalake utiliza una arquitectura plana para almacenar los datos.

\noindent \textbf{ABSTRACT}

\noindent \textbf{}

\noindent A business intelligence system is made up of different elements, but all the main parts of them are the data warehouse or the data warehouse, and the component that stores the data to be analyzed. In this chapter we will explain what a data warehouse is, how data is loaded in the data warehouse, extraction processes, Transformation and Loading (ETL) and how data warehouse data can be consulted easily, efficiently and powerful through OLAP technologies.

\noindent 

\noindent Surely you have heard the term Data Warehouse many times; We can define it as a corporate database where information is integrated and filtered from several different sources, which are then processed and analyzed as points of view with affinity of perspectives and great response speeds.

\noindent 

\noindent The creation of the Data Warehouse represents most of the time the first step, from the technical point of view, to implement a complete and reliable solution of Business Intelligence and thus also the best answers to the problems of the organization.

\noindent 

\noindent On the other hand, we see Datalake is a storage repository that contains a large amount of raw data and that stays there until the sea is necessary. A difference of a data store that stores data in folders or files, a data lake uses a flat architecture to store the data.

\noindent \textbf{}

\noindent \textbf{}

\noindent \textbf{}

\noindent \textbf{}

\noindent \textbf{}

\noindent \textbf{}

\noindent \textbf{}

\noindent \textbf{}

\begin{enumerate}
\item \textbf{ INTRODUCCI\'{O}N}
\end{enumerate}

\noindent 

\noindent Internet y las nuevas tecnolog\'{i}as han provocado el acceso y el almacenamiento desmesurado de informaci\'{o}n de los clientes y potenciales. Las empresas son cada vez m\'{a}s conscientes de la importancia que tienen esos datos para conocer mejor a los usuarios y as\'{i} poder ofrecerles aquello que realmente piden, y no lo que nosotros pensamos que necesitan. 

\noindent 

\noindent Esto es lo que se llama, aplicar estrategias customer centric. Para ello se necesita gestionar altos vol\'{u}menes de datos, tanto en tiempo real como organizados. Para ello, no hay nada mejor que un Data Warehouse o un Data Lake. Si no sabes exactamente en qu\'{e} consisten, no te preocupes, en este post te cuento de una manera sencilla, qu\'{e} son, para qu\'{e} sirven y las principales ventajas, ¿vamos a por ello?

\noindent 

\noindent El t\'{e}rmino de Data Warehouse fue acu\~{n}ado por Bill Inmon, traduci\'{e}ndose literalmente como Almac\'{e}n de Datos. Sin embargo, si fuera meramente un almac\'{e}n de datos, no solucionar\'{i}a el principal problema por el que se cre\'{o}, estructurar de una manera l\'{o}gica la informaci\'{o}n, con el objetivo de poder construir consultas que aporten informaci\'{o}n de valor al analista de datos.

\noindent 

\noindent 

\noindent 

\noindent 

\noindent 

\noindent 

\noindent 

\noindent 

\noindent \textbf{}

\noindent \textbf{}

\begin{enumerate}
\item \textbf{ MATERIALES Y METODOS}
\end{enumerate}

\noindent \textbf{}

\begin{enumerate}
\item \textbf{ Equipos, materiales, programas y recursos utilizados:}
\end{enumerate}

\noindent \textbf{}

\begin{enumerate}
\item \begin{enumerate}
\item \textbf{ }Computadora con sistema operativo Windows 8.1.
\end{enumerate}
\end{enumerate}

\noindent \textbf{}

\begin{enumerate}
\item \textbf{ MARCO TEORICO}
\end{enumerate}

\noindent \textbf{}

\begin{enumerate}
\item \begin{enumerate}
\item \textbf{ DATA WAREHOUSE}
\end{enumerate}
\end{enumerate}

\noindent \textbf{}

\noindent \textbf{Caracter\'{i}sticas:}

\begin{enumerate}
\item \textbf{ }Los datos almacenados en el Data Warehouse deben integrarse en una estructura consistente. La informaci\'{o}n, adem\'{a}s, debe estructurarse en diferentes niveles, adecu\'{a}ndose a las necesidades de cada uno de los usuarios.

\item  Los datos se deben de organizar por temas para facilitar su acceso y entendimiento por parte de los usuarios. Por ejemplo, todos los datos sobre ventas, deben de estar almacenados en el mismo sitio, de tal modo que, al realizar la consulta sobre ventas, sea m\'{a}s sencillo.

\item  Los datos suelen representar una situaci\'{o}n en un momento presente, sin embargo, el Data Warehouse debe de cargarse con los distintos valores que toma una variable en el tiempo para permitir analizar las tendencias y crear un hist\'{o}rico.

\item  La informaci\'{o}n que se almacena en un Data Warehouse es permanente y no debe ser modificada. Se deben de incorporar nuevos valores de las mismas variables, sin realizar ninguna acci\'{o}n sobre las ya existentes. De este modo podemos sacar conclusiones.
\end{enumerate}

\noindent 

\noindent Sin embargo, el objetivo \'{u}ltimo del Data Warehouse, no es otro que facilitar el procesamiento de datos, con el fin de analizar dicha informaci\'{o}n desde diferentes puntos de vista y a gran velocidad.

\noindent 

\noindent Para ello, es fundamental poder realizar un an\'{a}lisis multidimensional. De este modo, si queremos conocer el n\'{u}mero de ventas del modelo de zapatillas X, color azul, de la tienda de la calle Real, en La Coru\~{n}a, del a\~{n}o 2016 al a\~{n}o 2018, disponiendo de un Data Warehouse, el proceso es sencillo, ya que previamente hemos realizado una jerarquizaci\'{o}n de la informaci\'{o}n y creado diferentes dimensiones.

\noindent 

\noindent Otra caracter\'{i}stica importante del Data Warehouse, son los metadatos, ¿qu\'{e} es esto? Muy sencillo. Imag\'{i}nate que tienes una serie de datos almacenados, pero no sabes de d\'{o}nde proceden, cu\'{a}ndo se incluyeron, su fiabilidad, la forma de calcularlos{\dots} Con los metadatos tienes toda esa informaci\'{o}n. Estos metadatos son tambi\'{e}n los responsables de que se puedan construir consultas, informes o an\'{a}lisis.

\noindent 

\noindent \textbf{Principales ventajas del uso de un Data Warehouse}

\noindent Estas son las principales ventajas que se pueden encontrar en la implantaci\'{o}n de un Data Warehouse en el proceso de gesti\'{o}n del dato en tu negocio:

\noindent 

\begin{enumerate}
\item  Facilita la toma de decisiones basadas en datos, en cualquier \'{a}rea funcional de la empresa, ya que te proporciona informaci\'{o}n integrada y global del negocio.

\item  La informaci\'{o}n se convierte en un valor a\~{n}adido para cualquier negocio, gracias a que permite aplicar t\'{e}cnicas estad\'{i}sticas de an\'{a}lisis y modelizaci\'{o}n que ayudan a encontrar relaciones ocultas entre los datos almacenados.

\item  Te permite de manera sencilla aprender de los datos del pasado y predecir situaciones futuras para diferentes escenarios.

\item  Simplifica la implantaci\'{o}n de sistemas de gesti\'{o}n integral de la relaci\'{o}n con el cliente, dentro de la empresa.

\item  Supone una optimizaci\'{o}n tecnol\'{o}gica y econ\'{o}mica en entornos de Centro de Informaci\'{o}n, estad\'{i}stica o de generaci\'{o}n de informes con retornos de la inversi\'{o}n espectaculares.

\item  Es un sistema especialmente \'{u}til para el medio y el largo plazo.

\item  Aumenta la productividad de las empresas de manera muy sustancial.

\item  Te permite realizar planes de una manera mucho m\'{a}s efectiva.

\item  Permite la integraci\'{o}n de todas las herramientas corporativas. Por ejemplo, nosotros en Artyco integramos toda la informaci\'{o}n que recogemos a trav\'{e}s de todas nuestras aplicaciones (monitorizaci\'{o}n web, crm, wifi tracking, campa\~{n}as{\dots}) en un Data Warehouse, de donde sacar la informaci\'{o}n necesaria ante consultas determinadas.

\item  Para trabajar de manera correcta un Data Warehouse, es preciso que todos los componentes de la organizaci\'{o}n hablen el mismo lenguaje, es decir, que todos llamen a las cosas por su nombre. De este modo, gracias al Data Warehouse se pueden unificar conceptos.
\end{enumerate}

\noindent 

\noindent 

\begin{enumerate}
\item \begin{enumerate}
\item  \textbf{DATA LAKE}
\end{enumerate}
\end{enumerate}

\noindent \textbf{}

\noindent Un Data Lake no es otra cosa que un gran almac\'{e}n de datos en bruto, los cuales se mantienen tal cual han llegado, y hasta que se necesitan para su uso. La principal diferencia con el Data Warehouse, est\'{a} en la jerarqu\'{i}a y el almacenamiento de los datos en ficheros y carpetas que utiliza este, frente a la arquitectura plana del Data Lake. Podr\'{i}amos decir que el Data Lake se nutre de Big Data y datos en tiempo real, tanto estructurados como no estructurados, en una amalgama plana, sobre la cual puedes recoger aquella informaci\'{o}n que necesites.

\noindent \textbf{}

\noindent \textbf{Caracter\'{i}sticas}

\noindent \textbf{}

\begin{enumerate}
\item \textbf{ }Permite una f\'{a}cil y r\'{a}pida b\'{u}squeda de datos. El Data Lake est\'{a} asociado al Big Data, en el sentido de que es el recipiente donde descansan todos esos datos. Al no estar organizados como en el Data Warehouse, se hace necesaria una b\'{u}squeda eficiente de la informaci\'{o}n que en este se contiene. Esta b\'{u}squeda se realiza b\'{a}sicamente a trav\'{e}s de machine learning

\item  Un Data Lake inteligente permite analizar eficazmente el grado de protecci\'{o}n de la informaci\'{o}n que se guarda en los diferentes silos. Con la nueva normativa europea GDPR, esta seguridad en la privacidad de los datos se ve asegurada.

\item  El Data Lake te permite ser r\'{a}pido y disponer de datos en tiempo real. Adem\'{a}s, te permite preparar y compartir r\'{a}pidamente datos que son fundamentales para ofrecer anal\'{i}ticas competitivas.

\item  Te permite guardar pasos de preparaci\'{o}n de datos y luego reproducir r\'{a}pidamente esos pasos dentro de procesos automatizados. Es decir, muchas veces los analistas repiten las mismas actividades en la preparaci\'{o}n de datos. Con un Data Lake inteligente, puedes acceder a esos procesos y reducir tiempos y esfuerzos.
\end{enumerate}

\noindent \textbf{}

\noindent \textbf{Principales beneficios de un Data Lake}

\noindent Un Data Lake tiene muchas ventajas. Las m\'{a}s destacables son estas:

\noindent 

\begin{enumerate}
\item  El Data Lake permite centralizar todos los datos en un mismo lugar, vengan de la fuente que vengan. Una vez incluidas en su silo correspondiente de informaci\'{o}n, pueden ser procesadas a trav\'{e}s de herramientas de Big Data. Muchas veces, en esa disparidad de informaci\'{o}n, habr\'{a} datos que requieran un tratamiento especial en cuanto a seguridad. Gracias al Data Lake, este aspecto se puede solventar.

\item  Puede que la fuente original del dato est\'{e} obsoleta o se haya desactivado, sin embargo, su contenido puede que siga siendo valioso para el an\'{a}lisis. A trav\'{e}s del Data Lake, puedes acceder a dicha informaci\'{o}n.

\item  Todo dato que llegue al Data Lake puede ser normalizado y enriquecido.

\item  Los datos se preparan en funci\'{o}n de la necesidad del momento. Esto permite reducir considerablemente los costes y los tiempos. En el Data Warehouse, por ejemplo, es necesaria dicha preparaci\'{o}n.

\item  Se puede acceder a la informaci\'{o}n y enriquecerla desde cualquier punto del planeta, por cualquier usuario autorizado por el Data Lake. Esto ayuda a la organizaci\'{o}n a recopilar m\'{a}s f\'{a}cilmente los datos necesarios para la toma de decisiones.

\item  Un Data Lake pone la informaci\'{o}n en manos de un mayor n\'{u}mero de personas dentro de cualquier organizaci\'{o}n, aprovech\'{a}ndose mejor la empresa de ese conocimiento que adquieren dichos individuos.
\end{enumerate}

\noindent 

\begin{enumerate}
\item \begin{enumerate}
\item  \textbf{DIFERENCIAS ENTRE DATA WAREHOUSE Y DATA LAKE}
\end{enumerate}
\end{enumerate}

\noindent \textbf{}

\noindent \textbf{Podemos resumirlas en cinco grandes diferencias.}

\noindent \textbf{}

\begin{enumerate}
\item \textbf{ }Un Data Lake conserva todos los datos, no s\'{o}lo los que podr\'{i}an utilizarse actualmente, sino tambi\'{e}n aquello que podr\'{i}an necesitarse en un futuro. En frente, est\'{a} el Data Warehouse que estudia muy bien qu\'{e} datos incluir, cu\'{a}les son las fuentes de los datos. Adem\'{a}s, se necesita dedicar tiempo para entender el negocio y as\'{i} perfilar los datos. El Data Warehouse al final, contiene un modelo de datos altamente estructurado, dise\~{n}ado para la generaci\'{o}n de informes. El Data Lake utiliza un hardware muy diferente al del Data Warehouse. En el Data Lake, la ampliaci\'{o}n a terabytes y petabytes es mucho m\'{a}s econ\'{o}mico que en el caso del Data Warehouse. Es por eso, que en este \'{u}ltimo se mira tanto qu\'{e} datos son necesarios para cºonservar, y cuales eliminar, ya que supone un costoso almacenamiento.

\item  Un Data Lake soporta todos los tipos de datos, es decir, en este se guardan todos los datos, independientemente de la fuente y la estructura, y adem\'{a}s, se mantienen en su forma bruta, transform\'{a}ndolos s\'{o}lo cuando van a ser utilizados. En el Data Warehouse los datos almacenados son muchos m\'{a}s cr\'{i}ticos para el negocio y la realizaci\'{o}n de informes. Por ejemplo, los datos de im\'{a}genes, comentarios en redes sociales, textos, etc, no suelen ser tenidos en cuenta, ya que, como he comentado, su almacenamiento es muy costoso.

\item  Los Data Lakes son m\'{a}s flexibles que los Data Warehouses. Uno de los mayores problemas que presenta un Data Warehouse, est\'{a} en el momento que se necesita realizar un cambio importante. Todo cambio se convierte en una tarea realmente dif\'{i}cil, ya que adaptar un Data Warehouse supone invertir mucho tiempo en el desarrollo de la estructura del almac\'{e}n. Hoy d\'{i}a, las organizaciones demandan respuestas r\'{a}pidas a sus preguntas comerciales, y en muchos casos, no pueden esperar a que el Data Warehouse se adapte. En cambio, el Data Lake, al almacenar todos los datos en bruto, permite el acceso de cualquier usuario para que los explote y analice en funci\'{o}n de sus necesidades, encontrando la manera de responder a sus preguntas a su ritmo.

\item  El Data Warehouse te proporciona unos resultados m\'{a}s limpios, estructurados y fiables. Sin embargo, en el Data Lake, al disponer de datos en bruto y sin estructurar, al hacer las consultas, usuarios no demasiado cualificados, recibir\'{a}n informaci\'{o}n r\'{a}pida, pero no del todo precisa, tal y como la obtendr\'{i}an de un Data Warehouse. Normalmente, para usuarios de perfil Data scientist, este problema no existe en el Data Lake, ya que ellos crean sus reglas y estructuran la informaci\'{o}n para preparar sus an\'{a}lisis y modelos. El verdadero problema reside en el 80\% del resto de usuarios, quienes simplemente buscan tener acceso a ciertos kpis diarios.
\end{enumerate}

\noindent 

\noindent Tanto los Data Warehouses como los Data Lakes est\'{a}n destinados a convivir en las empresas que deseen basar sus decisiones en datos. Como habr\'{a}s podido entender, ambos son complementarios, no sustitutivos, pudiendo ayudar a cualquier negocio a conocer mejor el mercado y el consumidor, de cara a poder realizar estrategias basadas en el conocimiento de estos, con comunicaciones cada vez m\'{a}s personalizadas, es decir, ser m\'{a}s customer centric.

\noindent 

\noindent 

\noindent \includegraphics*[width=6.10in, height=3.28in, keepaspectratio=false]{Imagen2}

\noindent 

\noindent 

\noindent \includegraphics*[width=6.10in, height=2.75in, keepaspectratio=false]{Imagen3}

\begin{enumerate}
\item  \textbf{CONCLUSIONES}
\end{enumerate}

\noindent \textbf{}

\begin{enumerate}
\item \textbf{ }El concepto de Data Warehouse abarca mucho m\'{a}s que simplemente copiar datos operacionales a una base de datos informacional distinta. El sistema deber\'{a} ofrecer una soluci\'{o}n completa para gestionar y controlar el flujo de informaci\'{o}n desde bases de datos corporativas y fuentes externas a sistemas de soporte de decisiones de usuarios finales. Adem\'{a}s, debe permitir a los usuarios conocer qu\'{e} informaci\'{o}n existe en el almac\'{e}n de datos, y c\'{o}mo poder acceder a ella y manipularla.
\end{enumerate}

\noindent 

\begin{enumerate}
\item  La discusi\'{o}n Data Lake vs Data Warehouse es algo muy com\'{u}n entre aquellas empresas que se disponen a implantar soluciones de big data. R\'{a}pidamente la conversaci\'{o}n sobre datos y an\'{a}lisis en el \'{a}mbito de big data nos lleva al Data Lake o lago de datos, pero muy a menudo las empresas no acaban de entender bien qu\'{e} es lo que esto significa y cu\'{a}les son las diferencias entre Data Lake vs Data Warehouse.
\end{enumerate}

\noindent 

\begin{enumerate}
\item  De hecho, se pueden complementar muy bien, dise\~{n}ando una arquitectura de datos moderna, que permita seguir a las organizaciones aprovechando sus inversiones en su Data Warehouse, mientras que empiezan a recoger en su Data Lake, todos los datos que han sido ignorados o desechados anteriormente.
\end{enumerate}

\noindent 

\noindent 

\noindent 

\noindent \textbf{}

\begin{enumerate}
\item \textbf{ RECOMENDACIONES }
\end{enumerate}

\noindent \textbf{}

\begin{enumerate}
\item \textbf{ }Debe enfocarse a toda la empresa: debe proveer informaci\'{o}n de todas las \'{a}reas de la empresa como ventas, finanzas, operaci\'{o}n, etc.
\end{enumerate}

\noindent 

\begin{enumerate}
\item  El dise\~{n}o debe ajustarse a los cambios: es decir que debe poder adaptarse al cambio de las reglas de negocio, permitiendo que el Data Warehouse debe seguir brindando informaci\'{o}n de soporte a decisiones en un nuevo contexto de las necesidades de la empresa.
\end{enumerate}

\noindent 

\begin{enumerate}
\item  Preparado para carga masiva de datos: El proceso de carga no debe ser extenso en complejidad y tiempo de ejecuci\'{o}n, se deben aplicar todas las t\'{e}cnicas de gesti\'{o}n especializada de datos de manera de tener un proceso eficiente. Recordemos que estos sistemas est\'{a}n dise\~{n}ados preferentemente para el an\'{a}lisis de la informaci\'{o}n, es de decir, la obtenci\'{o}n de indicadores o KPI.
\end{enumerate}

\noindent 

\begin{enumerate}
\item  Debe ser multiprop\'{o}sito: Su dise\~{n}o debe ser f\'{a}cil de entender, de escalar y de actualizar, de manera de estar preparado para todas las formas posibles an\'{a}lisis de Business Intelligence.
\end{enumerate}

\noindent \textbf{}

\noindent \textbf{}

\noindent \textbf{}

\noindent \textbf{}

\noindent \textbf{}

\noindent \textbf{}

\noindent \textbf{}

\noindent \textbf{}

\begin{enumerate}
\item \textbf{ BIBLIOGRAF\'{I}A}
\end{enumerate}

\noindent \textbf{}

\begin{enumerate}
\item \textbf{ }https://blog.powerdata.es/el-valor-de-la-gestion-de-datos/data-lake-vs-data-warehouse.-veamos-sus-principales-diferencias
\end{enumerate}

\noindent 

\begin{enumerate}
\item  https://tableauperu.com/data-warehouse/
\end{enumerate}

\noindent 

\begin{enumerate}
\item  https://blog.powerdata.es/el-valor-de-la-gestion-de-datos/data-lake-vs-data-warehouse.-veamos-sus-principales-diferencias
\end{enumerate}

\noindent 

\begin{enumerate}
\item  https://trends.inycom.es/principales-diferencias-data-lakes-data-warehouse/
\end{enumerate}

\noindent 

\begin{enumerate}
\item  pg 21 Thank you! See you Thursday, March 2 for the next webinar, Descriptive, Prescriptive and Predictive Analytics John Ladley @jladley john@firstsanfranciscopartners.com 
\end{enumerate}

\noindent 

\begin{enumerate}
\item  KO'Neal @kellezoneal kelle@firstsanfranciscopartners.com {\copyright} 2016 First San Francisco Partners www.firstsanfranciscopartners.com
\end{enumerate}

\noindent \textbf{}

\noindent 


\end{document}

